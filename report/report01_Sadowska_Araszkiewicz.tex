\documentclass[a4paper 10pt]{article}
\usepackage[english,polish]{babel}
\usepackage[MeX]{polski}
\usepackage[utf8]{inputenc}
\usepackage[T1]{fontenc}
\usepackage[letterpaper, landscape, margin=1.7in]{geometry}
%\usepackage {amsthm}
\usepackage{graphicx}
%\usepackage{mathrsfs}
%\usepackage{hyperref}
\usepackage{listings}
\usepackage{subfigure}
\usepackage{dashrule}
\usepackage{listings}
\usepackage{float}
%\usepackage{indentfirst}
\usepackage{amsmath}
\usepackage{listings}
\usepackage{multirow}
\usepackage{amsmath}
\usepackage{xcolor}
\usepackage{listings}
\lstset{
    frame=single,
    breaklines=true,
    postbreak=\raisebox{0ex}[0ex][0ex]{\ensuremath{\color{red}\hookrightarrow\space}}
}


\renewcommand{\rmdefault}{ptm}
  
\frenchspacing

% Used to add additional dot in enumerations
\usepackage{titlesec}
\titlelabel{\thetitle.\quad}
\title{\textbf{Techniki Optymalizacji} \\
Labolatorium nr 1 \\
Sprawozdanie}
\author{Paulina Sadowska, Rafał Araszkiewicz}
\begin{document}
\maketitle

\section{Wprowadzenie}
Celem ćwiczenia było zaimplementowanie algorytmów rozwiazujących problem Komiwojażera dla zbioru 100 punktów. Algorytmy te znaleźc miały najbardziej optymalną ścieżkę łączącą 50 dowolnych punktów grafu gdzie punktem startowym miał być każdy z punktów znujdujacych się w zbiorze.
\section{Nearest Neighbour}
\subsection{Opis}
\subsection{Implementacja w pseudokodzie}
\subsection{Wyniki}
\begin{table}[H]
\center
\caption{Greedy Cycle - wyniki}
\label{Greedy Cycle - wyniki}
\begin{tabular}{|p{1cm}|p{14cm}|}
\hline
min       & 10298 \\ \hline
mean      & 12793                                                                                                                                                                                                 \\ \hline
max       & 14877 \\ \hline
best path & 80, 24, 60, 50, 86, 8, 6, 56, 19, 11, 26, 85, 34, 61, 59, 76, 22, 97, 90, 44, 31, 10, 14, 16, 73, 20, 58, 71, 9, 83, 35, 37, 23, 17, 78, 52, 87, 15, 21, 93, 69, 65, 64, 3, 96, 55, 79, 30, 88, 41, 80 \\ \hline
\end{tabular}
\end{table}

\section{Greedy Cycle}
\label{Greedy}
\subsection{Opis}
W algorytmie tym trasa jest budowana w taki sposób, aby zawsze tworzyła cykl Hamiltona. W każdej iteracji dodawany jest jeden najkrótszy łuk z pozostałych dostępnych.
\subsection{Implementacja w pseudokodzie}
\begin{lstlisting}[frame=single]
dla kazdego z zdefiniowanych punktow poczatkowych
	dodaj do sciezki punkt poczatkowy
	dla kazdego z punktow oprocz poczatkowego
	koszt = odleglosc pomiedzy tym punktem a poczatkowym
		jezeli koszt < dotychczasowy najmniejszy koszt
			dotychczasowy najmniejszy koszt = koszt
	koniec
	
	dodaj do sciezki punkt o najmniejszym koszcie (odleglosci)
	dodaj na koncu sciezki punkt poczatkowy
	
	dla pozostalych 48 krokow
		dla wszystkich par punktow w sciezce
			dla wszystkich nieodwiedzonych punktow
				koszt = odleglosc miedzy para punktow - odleglosc miedzy trojka punktow
				jezeli koszt < dotychczasowy najmniejszy koszt
					dotychczasowy najmniejszy koszt = koszt
			koniec
		koniec
		dodaj punkt powodujacy najmniejsza zmiane kosztu w odpowiednie miejsce w sciezce
	koniec	
koniec

\end{lstlisting}
\subsection{Wyniki}
\begin{table}[H]
\center
\caption{Greedy Cycle - wyniki}
\label{Greedy Cycle - wyniki}
\begin{tabular}{|p{1cm}|p{14cm}|}
\hline
min       & 11095                                                                                                                                                                                                 \\ \hline
mean      & 12653                                                                                                                                                                                                 \\ \hline
max       & 13393                                                                                                                                                                                                 \\ \hline
best path & 61, 34, 85, 26, 11, 19, 6, 8, 56, 86, 50, 24, 80, 60, 57, 66, 27, 92, 0, 7, 91, 74, 96, 18, 52, 15, 69, 21, 93, 87, 17, 23, 37, 83, 78, 89, 48, 5, 62, 46, 10, 16, 14, 31, 44, 90, 97, 22, 76, 59, 61 \\ \hline
\end{tabular}
\end{table}

\section{Nearest Neighbour Grasp}
\subsection{Opis}
\subsection{Implementacja w pseudokodzie}
\subsection{Wyniki}

\section{Greedy Cycle Grasp}
\subsection{Opis}
Modyfikacja algorytmu Greedy Cycle opisanego w punkcie \ref{Greedy} polegająca na tym, że w każdym kroku szukamy 3 najlepszych rozwiazań i z nich losulemy te, które będzie dodane do ścieżki.
\subsection{Implementacja w pseudokodzie}
\subsection{Wyniki}

\end{document}