\documentclass[a4paper 10pt]{article}
\usepackage[english,polish]{babel}
\usepackage[MeX]{polski}
\usepackage[utf8]{inputenc}
\usepackage[T1]{fontenc}
\usepackage{times}
%\usepackage {amsthm}
\usepackage{graphicx}
%\usepackage{mathrsfs}
%\usepackage{hyperref}
\usepackage{listings}
\usepackage{subfigure}
\usepackage{dashrule}
\usepackage{listings}
\usepackage{float}
%\usepackage{indentfirst}
\usepackage{amsmath}
\usepackage{listings}
\usepackage{multirow}



\renewcommand{\rmdefault}{ptm}
  
\frenchspacing

% Used to add additional dot in enumerations
\usepackage{titlesec}
\titlelabel{\thetitle.\quad}
\title{\textbf{Techniki Optymalizacji} \\
Labolatorium nr 1 \\
Sprawozdanie}
\author{Paulina Sadowska, Rafał Araszkiewicz}

\lstset{language=Matlab}
\begin{document}
\maketitle

\section{Wprowadzenie}
Celem ćwiczenia było zaimplementowanie algorytmów rozwiazujących problem Komiwojażera dla zbioru 100 punktów. Algorytmy te znaleźc miały najbardziej optymalną ścieżkę łączącą 50 dowolnych punktów grafu gdzie punktem startowym miał być każdy z punktów znujdujacych się w zbiorze.
\section{Nearest Neighbour}
\subsection{Opis}
\subsection{Implementacja w pseudokodzie}
\subsection{Wyniki}

\section{Greedy Cycle}
\subsection{Opis}
\subsection{Implementacja w pseudokodzie}
\subsection{Wyniki}

\section{Nearest Neighbour Grasp}
\subsection{Opis}
\subsection{Implementacja w pseudokodzie}
\subsection{Wyniki}

\section{Greedy Cycle Grasp}
\subsection{Opis}
\subsection{Implementacja w pseudokodzie}
\subsection{Wyniki}

\end{document}